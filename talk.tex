\documentclass{beamer}
\usepackage[utf8]{inputenc}
\usepackage{amsmath}
\usepackage{minted}
\usepackage{pgf}
\usepackage{tikz}
\usepackage{upquote}
\usepackage{hyperref}
\usepackage{graphicx}
\usetikzlibrary{arrows,automata,positioning}

\setbeamertemplate{footline}[frame number]
\setbeamertemplate{navigation symbols}{}

\title{Practical examples of writing programs and proving theorems in Idris.}
\author{Donovan Crichton}
\date{January 2020}

\begin{document}
 
\frame{\titlepage}

\begin{frame}[fragile]
  \frametitle{Preliminaries}

\end{frame}

\begin{frame}[fragile]
  \frametitle{Propositional Logic}
  \begin{itemize}
    \item Concerned with statements of verifiable facts. 
    \item Used daily by programmers when reasoning about Boolean values.
  \end{itemize}
  \begin{tabular}{| c | c | c |}
  \hline
  \textbf{Symbol} & \textbf{Meaning} & \textbf{Example} \\
  \hline
  T, F & True, False & Boolean values. \\
  $p, q, r, ...$  & Propositions & Let $p$ = "It is raining." \\
  $\lnot$ & Negation (Not) & $\lnot p$ \\
  $\land$ & Conjuction (And) & $p \land q$ \\
  $\lor$ & Disjunction (Or) & $p \lor q$ \\
  $\rightarrow$ & Implication (If) & $p \rightarrow q$ \\
  $\leftrightarrow$ & Bi Implication (Iff) & $p \leftrightarrow q$ \\
  $\equiv$ & Equivalence & $p \equiv q$ \\
  $\top$ & Tautology & $p \lor \lnot p \equiv \top$ \\
  $\bot$ & Contradiction & $p \land \lnot p \equiv \bot$ \\
  \hline
  \end{tabular}
\end{frame}

\begin{frame}[fragile]
  \frametitle{Definitions of Connectives}
  \begin{minipage}{0.32\textwidth}
    Conjunction (And) \\
    \begin{tabular}{| c | c | c |}
    \hline
    $p$ & $q$ & $p \land q$ \\
    \hline
    T & T & T \\
    T & F & F \\
    F & T & F \\
    F & F & F \\
    \hline
    \end{tabular}
  \end{minipage}
  \begin{minipage}{0.32\textwidth}
    Disjunction (Or) \\
    \begin{tabular}{| c | c | c |}
    \hline
    $p$ & $q$ & $p \lor q$ \\
    \hline
    T & T & T \\
    T & F & T \\
    F & T & T \\
    F & F & F \\
    \hline
    \end{tabular}
  \end{minipage}
  \begin{minipage}{0.32\textwidth}
    Negation (Not) \\
    \begin{tabular}{| c | c |}
    \hline
    $p$ & $\lnot p$ \\
    \hline
    T & F \\
    F & T \\
    \hline
    \end{tabular}
  \end{minipage}
  \begin{minipage}{0.32\textwidth}
    $\;$ \\
    Implication (If) \\
    \begin{tabular}{| c | c | c |}
    \hline
    $p$ & $q$ & $p \rightarrow q$ \\
    \hline
    T & T & T \\
    T & F & F \\
    F & T & T \\
    F & F & T \\
    \hline
    \end{tabular}
  \end{minipage}
  \begin{minipage}{0.32\textwidth}
    $\;$ \\
    Bi Implication (Iff) \\
    \begin{tabular}{| c | c | c |}
    \hline
    $p$ & $q$ & $p \leftrightarrow q$ \\
    \hline
    T & T & T \\
    T & F & F \\
    F & T & F \\
    F & F & T \\
    \hline
    \end{tabular}
  \end{minipage}
  \begin{minipage}{0.32\textwidth}
    $\;$ \\
    Logical Equivalence \\
    \begin{tabular}{| c | c | c |}
    \hline
    $p$ & $q$ & $p \equiv q$ \\
    \hline
    T & T & T \\
    T & F & F \\
    F & T & F \\
    F & F & T \\
    \hline
    \end{tabular}
  \end{minipage}
\end{frame}

\begin{frame}[fragile]
  \frametitle{Proof Techniques}
  \framesubtitle{By Exhaustion}
  Idea: Prove by enumerating all possible cases. \\
  Prove: $(\lnot p \lor q) \leftrightarrow (p \rightarrow q)$.
  \begin{tabular}{| c | c | c | c | c | c |}
  \hline
  $p$ & $q$ & $\lnot p$ & $ \lnot p \lor q$ & $p \rightarrow q$
      & $(\lnot p \lor q) \leftrightarrow (p \rightarrow q)$ \\
  \hline
  T & T & F & T & T & T \\
  T & F & F & F & F & T \\
  F & T & T & T & T & T \\
  F & F & T & T & T & T \\
  \hline
  \end{tabular}
\end{frame}

\begin{frame}[fragile]
  \frametitle{Proof Techniques}
  \framesubtitle{By Appeal to Lemma}
  Idea: Introduce pre-proven smaller proofs (called a Lemma) to prove a larger
        proof. \\
  \begin{itemize}
    \item \textbf{Lemma 1.} $p \lor \lnot p \equiv \top$. \\
    \item \textbf{Lemma 2.} $(p \equiv q) \equiv (p \leftrightarrow q)$. 
  \end{itemize} 
  Prove: $(p \leftrightarrow q) \lor \lnot (p \equiv q) \leftrightarrow \top$.
  \begin{align*}
    (p \leftrightarrow q) \lor \lnot (p \equiv q) &\leftrightarrow \top
          &\text{Premise.} \\
    (p \equiv q) \lor \lnot (p \equiv q) &\equiv \top
          &\text{Lemma 2.} \\
    \top &\equiv \top &\text{Lemma 1.} \\
    & &\qed
  \end{align*}
\end{frame}

\begin{frame}
  \frametitle{First Order (or Predicate) Logic}
  Extends propositional logic from reasoning about propositions to reasoning
  about sets. \\
  \begin{tabular}{| c | c | c |}
  \hline
  \textbf{Symbol} & \textbf{Meaning} & \textbf{Example} \\
  \hline
  $X, Y, Z, ...$ & Set Variables & Let $Y$ = $\{2, 3, 4\}$. \\
  $x, y, z, ...$ & Member Variables & Let $z$ = 2.\\
  $P(x), Q(y), ...$ & Predicate Variables & Let $Q(y)$ = $y > 1$ .\\
  $\forall x \in X, P(x)$ & Universal Quantifier & $\forall y \in Y, Q(y)$ \\
  $\exists x \in X, P(x)$ & Existential Quantifier & $\exists z \in Y, z = 2$\\
  \hline
  \end{tabular}
\end{frame}

\begin{frame}
  \frametitle{Proof Techniques}
  \framesubtitle{Induction}
  \begin{itemize}
    \item Allows us to prove that a property $P(x)$ holds $\forall x \in X$.
          Provided $X$ is well-founded.
    \item Informally well-founded means ``no infinite decreasing chains''.
  \end{itemize}
   Prove: $\forall n \in \mathbb{N}(\exists y \in \mathbb{N}, y = n + 1)$ \\
   \begin{align*}
   y &= 0 + 1 &\text{Base Case. $n = 0$} \\
     &= 1 &\qed \\
   \\
   y &= (k + 1) + 1 &\text{Inductive Step. $n = k + 1$} \\
     &= k + 2 &\qed
   \end{align*}
\end{frame}


\begin{frame}
  \frametitle{Why should I care?}
  \begin{itemize}
    \item Types are just sets with flavour!
    \item \textbf{Bool} = $\{True, False\}$
    \item \textbf{Int} = $\{-\infty, ..., -2, -1, 0, 1, 2, 3, ..., \infty\}$
    \item Mixing of flavours is not allowed!
    \item $\{True, -2, "Hello", 1\}$ Can really only be said to be a "thing"
          flavoured set.
  \end{itemize}
\end{frame}

\begin{frame}
  \frametitle{Propositions as Types. Proofs as Programs}
  \begin{itemize}
    \item The Curry-Howard-Lambeck correspondence is well known amongst
          Haskell programmers for the correspondence between categories and
          programming.
    \item The correspondence to logic is less often discussed.
    \item Holds for any language that is based on a typed lambda calculus.
  \end{itemize}
  Idea: A type is a proposition. This is True if, and only if, we can construct
        an element of that type.
\end{frame}

\begin{frame}
  \frametitle{What is truth?}
\end{frame}

\begin{frame}
  \frametitle{Curry-Howard in Idris}
\end{frame}


\begin{frame}
  \frametitle{Vacuous Truths}
  Statements that are true because the premise cannot be satisfied. \\
  Often arise as a base case in proofs by induction.
  \begin{itemize}
    \item $F \rightarrow q \equiv T$
    \item $\forall x \in \{\} P(x)$.
  \end{itemize}
\end{frame}

\begin{frame}{Natural Numbers}
\end{frame}

\begin{frame}
  \frametitle{Some small examples in Idris}
\end{frame}



\end{document}
