\documentclass[usenames,dvipsnames]{beamer}
\usepackage[utf8]{inputenc}
\usepackage{amsmath}
\usepackage{minted}
\usepackage{pgf}
\usepackage{tikz}
\usepackage{upquote}
\usepackage{hyperref}
\usepackage{graphicx}
\usetikzlibrary{arrows,automata,positioning}

\setbeamertemplate{footline}[frame number]
\setbeamertemplate{navigation symbols}{}

\title{Practical examples of writing programs and proving theorems in Idris.}
\author{Donovan Crichton}
\date{January 2020}

\begin{document}
 
\frame{\titlepage}

\begin{frame}[fragile]
  \frametitle{Preliminaries}

\end{frame}

\begin{frame}[fragile]
  \frametitle{Propositional Logic}
  \begin{itemize}
    \item Concerned with statements of verifiable facts. 
    \item Used daily by programmers when reasoning about Boolean values.
  \end{itemize}
  \begin{tabular}{| c | c | c |}
  \hline
  \textbf{Symbol} & \textbf{Meaning} & \textbf{Example} \\
  \hline
  T, F & True, False & Boolean values. \\
  $p, q, r, ...$  & Propositions & Let $p$ = "It is raining." \\
  $\lnot$ & Negation (Not) & $\lnot p$ \\
  $\land$ & Conjuction (And) & $p \land q$ \\
  $\lor$ & Disjunction (Or) & $p \lor q$ \\
  $\rightarrow$ & Implication (If) & $p \rightarrow q$ \\
  $\leftrightarrow$ & Bi Implication (Iff) & $p \leftrightarrow q$ \\
  $\equiv$ & Equivalence & $p \equiv q$ \\
  $\top$ & Tautology & $p \lor \lnot p \equiv \top$ \\
  $\bot$ & Contradiction & $p \land \lnot p \equiv \bot$ \\
  \hline
  \end{tabular}
\end{frame}

\begin{frame}[fragile]
  \frametitle{Definitions of Connectives}
  \begin{minipage}{0.32\textwidth}
    Conjunction (And) \\
    \begin{tabular}{| c | c | c |}
    \hline
    $p$ & $q$ & $p \land q$ \\
    \hline
    T & T & T \\
    T & F & F \\
    F & T & F \\
    F & F & F \\
    \hline
    \end{tabular}
  \end{minipage}
  \begin{minipage}{0.32\textwidth}
    Disjunction (Or) \\
    \begin{tabular}{| c | c | c |}
    \hline
    $p$ & $q$ & $p \lor q$ \\
    \hline
    T & T & T \\
    T & F & T \\
    F & T & T \\
    F & F & F \\
    \hline
    \end{tabular}
  \end{minipage}
  \begin{minipage}{0.32\textwidth}
    Negation (Not) \\
    \begin{tabular}{| c | c |}
    \hline
    $p$ & $\lnot p$ \\
    \hline
    T & F \\
    F & T \\
    \hline
    \end{tabular}
  \end{minipage}
  \begin{minipage}{0.32\textwidth}
    $\;$ \\
    Implication (If) \\
    \begin{tabular}{| c | c | c |}
    \hline
    $p$ & $q$ & $p \rightarrow q$ \\
    \hline
    T & T & T \\
    T & F & F \\
    F & T & T \\
    F & F & T \\
    \hline
    \end{tabular}
  \end{minipage}
  \begin{minipage}{0.32\textwidth}
    $\;$ \\
    Bi Implication (Iff) \\
    \begin{tabular}{| c | c | c |}
    \hline
    $p$ & $q$ & $p \leftrightarrow q$ \\
    \hline
    T & T & T \\
    T & F & F \\
    F & T & F \\
    F & F & T \\
    \hline
    \end{tabular}
  \end{minipage}
  \begin{minipage}{0.32\textwidth}
    $\;$ \\
    Logical Equivalence \\
    \begin{tabular}{| c | c | c |}
    \hline
    $p$ & $q$ & $p \equiv q$ \\
    \hline
    T & T & T \\
    T & F & F \\
    F & T & F \\
    F & F & T \\
    \hline
    \end{tabular}
  \end{minipage}
\end{frame}

\begin{frame}[fragile]
  \frametitle{Proof Techniques}
  \framesubtitle{By Exhaustion}
  Idea: Prove by enumerating all possible cases. \\
  Prove: $(\lnot p \lor q) \leftrightarrow (p \rightarrow q)$.
  \begin{tabular}{| c | c | c | c | c | c |}
  \hline
  $p$ & $q$ & $\lnot p$ & $ \lnot p \lor q$ & $p \rightarrow q$
      & $(\lnot p \lor q) \leftrightarrow (p \rightarrow q)$ \\
  \hline
  T & T & F & T & T & T \\
  T & F & F & F & F & T \\
  F & T & T & T & T & T \\
  F & F & T & T & T & T \\
  \hline
  \end{tabular}
\end{frame}

\begin{frame}[fragile]
  \frametitle{Proof Techniques}
  \framesubtitle{By Appeal to Lemma}
  Idea: Introduce pre-proven smaller proofs (called a Lemma) to prove a larger
        proof. \\
  \begin{itemize}
    \item \textbf{Lemma 1.} $p \lor \lnot p \equiv \top$. \\
    \item \textbf{Lemma 2.} $(p \equiv q) \equiv (p \leftrightarrow q)$. 
  \end{itemize} 
  Prove: $(p \leftrightarrow q) \lor \lnot (p \equiv q) \leftrightarrow \top$.
  \begin{align*}
    (p \leftrightarrow q) \lor \lnot (p \equiv q) &\leftrightarrow \top
          &\text{Premise.} \\
    (p \equiv q) \lor \lnot (p \equiv q) &\equiv \top
          &\text{Lemma 2.} \\
    \top &\equiv \top &\text{Lemma 1.} \\
    & &\qed
  \end{align*}
\end{frame}

\begin{frame}[fragile]
  \frametitle{First Order (or Predicate) Logic}
  Extends propositional logic from reasoning about propositions to reasoning
  about sets. \\
  \begin{tabular}{| c | c | c |}
  \hline
  \textbf{Symbol} & \textbf{Meaning} & \textbf{Example} \\
  \hline
  $X, Y, Z, ...$ & Set Variables & Let $Y$ = $\{2, 3, 4\}$. \\
  $x, y, z, ...$ & Member Variables & Let $z$ = 2.\\
  $P(x), Q(y), ...$ & Predicate Variables & Let $Q(y)$ = $y > 1$ .\\
  $\forall x \in X, P(x)$ & Universal Quantifier & $\forall y \in Y, Q(y)$ \\
  $\exists x \in X, P(x)$ & Existential Quantifier & $\exists z \in Y, z = 2$\\
  \hline
  \end{tabular}
\end{frame}

\begin{frame}[fragile]
  \frametitle{Proof Techniques}
  \framesubtitle{Induction}
  \begin{itemize}
    \item Allows us to prove that a property $P(x)$ holds $\forall x \in X$.
          Provided $X$ is well-founded.
    \item Informally well-founded means ``no infinite decreasing chains''.
  \end{itemize}
   Prove: $\forall n \in \mathbb{N}(\exists y \in \mathbb{N}, y = n + 1)$ \\
   \begin{align*}
   y &= 0 + 1 &\text{Base Case. $n = 0$} \\
     &= 1 &\qed \\
   \\
   y &= (k + 1) + 1 &\text{Inductive Step. $n = k + 1$} \\
     &= k + 2 &\qed
   \end{align*}
\end{frame}


\begin{frame}[fragile]
  \frametitle{Why should I care?}
  \begin{itemize}
    \item Types are just sets with flavour!
    \item \textbf{Bool} = $\{True, False\}$
    \item \textbf{Int} = $\{-\infty, ..., -2, -1, 0, 1, 2, 3, ..., \infty\}$
    \item Mixing of flavours is not allowed!
    \item $\{True, -2, "Hello", 1\}$ Can really only be said to be a "thing"
          flavoured set.
  \end{itemize}
\end{frame}

\begin{frame}[fragile]
  \frametitle{Propositions as Types. Proofs as Programs}
  \begin{itemize}
    \item The Curry-Howard-Lambeck correspondence is well known amongst
          Haskell programmers for the correspondence between categories and
          programming.
    \item The correspondence with logic is less often discussed.
    \item Holds for any language that is based on a typed lambda calculus.
  \end{itemize}
  Idea: A type is a propostion.
\end{frame}

\begin{frame}[fragile]
  \frametitle{What is Truth?}
    Propositional Logic and Predicate Logic consider truth to be the Boolean
     value ``True''. These logics also have a notion of vacuous truth. \\
    $\;$ \\
    \begin{minipage}{0.32\textwidth}
    \begin{tabular}{| c | c | c |}
    \hline
    $p$ & $q$ & $p \rightarrow q$ \\
    \hline
    T & T & T \\
    T & F & F \\
    F & T & T \\
    F & F & T \\
    \hline
    \end{tabular}
    \end{minipage}
  $\;$\\ \\
  In Predicate Logic: $\forall x \in \{\} P(x)$ is also true.
  \begin{itemize}
   \item If a type is a proposition, what does it mean for it to be true?
   \item A type is true iff it is inhabited with a value.
  \end{itemize}
\end{frame}

\begin{frame}[fragile]
  \frametitle{Curry-Howard in Idris}
    \begin{tabular}{c|c|c|c}
    \textbf{Logic Term} & \textbf{Logic Symbol} & \textbf{Idris Symbol}
      & \textbf{Idris Type} \\
    \hline
      Implication & p $\Rightarrow$ q & p \mintinline{idris}{->} q
      & Arrow \\
      Conjunction & p $\land$ q & \mintinline{idris}{(p, q)}
      & Pair (Product) \\
      Disjunction & p $\lor$ q & \mintinline{idris}{Either p q}
      & Enum (Sum)\\
      Negation & $\lnot$ p & \mintinline{idris}{p -> Void} &
      Void Type \\
      IFF/Eq & p $\Leftrightarrow$ q, p $\equiv$ q & \mintinline{idris}{(p -> q, q -> p)}
      & Pair Arrows \\
      Universal & $\forall$ x. P x &
      \mintinline{idris}{p -> Type} & $\Pi$ Type \\
      Existential & $\exists$ x. P x
      & \mintinline{idris}{(x ** P x)} & $\Sigma$ Type \\
      \hline
            = & = & \mintinline{idris}{p = q} & Type Equality \\
            $\top$ & True & \mintinline{idris}{()} & Unit Type \\
            $\bot$ & False & \mintinline{idris}{Void} & Uninhabited
    \end{tabular}
\end{frame}

\begin{frame}[fragile]
\frametitle{Natural Numbers}
Let $\mathbb{N}$ denote the set of natural numbers where: \\
1. Zero (0) is a natural number. \\
2. If $k$ is a natural number, then the successor of $k$ is also a natural
        number.

\begin{minted}[escapeinside=||]{idris}
data Nat : Type where
  Z : Nat
  S : (k : Nat) -> Nat
\end{minted}
\begin{itemize}
  \item \textbf{Nat : Type} is the type constructor.
  \item \textbf{S} and \textbf{K} are the value constructors.
\end{itemize}
\end{frame}

\begin{frame}[fragile]
\end{frame}

\begin{frame}[fragile]
  \frametitle{Proving commutativity on addition in Idris}
  \begin{minted}[escapeinside=||]{idris}
  (+) : Nat -> Nat -> Nat
  Z     + y = Z
  (S k) + y = S (k + y)

  |\textcolor{olive}{$\forall x, y \in \mathbb{N}. x + y = y + x$}|
  plusIsCommutative : (x, y : Nat) -> x + y = y + x
  plusIsCommutative x y = ?what |\textcolor{red}{$\frac{x, y : Nat}{what : x + y = y + x}$}|
  
  plusIsCommutative : (x, y : Nat) -> x + y = y + x
  plusIsCommutative Z y = ?t1 |\textcolor{red}{$\frac{x, y : Nat}{t1 : y = y + Z}$}|
  plusIsCommutative (S k) y = ?t2 |\textcolor{red}{$\frac{x, y : Nat}{t2 : S(k + y) = y + (S\;k)}$}|
  \end{minted}
\end{frame}

\begin{frame}[fragile]
  \frametitle{Introducing Lemmas in Idris}
  \begin{minted}[escapeinside=||]{idris}
  |\textcolor{red}{$\frac{x, y : Nat}{t1 : y = y + Z}$}|
  lemma1 : (x : Nat) -> x = x + Z
  lemma1 Z = ?t1
  \end{minted}
\end{frame}


\end{document}
